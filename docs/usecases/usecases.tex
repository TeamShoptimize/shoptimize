\documentclass[a4paper,11pt]{scrartcl}
\usepackage[left=2.5cm, right=3cm]{geometry}              % flexible and complete interface to document dimensions
\usepackage[latin1]{inputenc}                             % font encoding for umlauts
\usepackage[ngerman]{babel}                               % language settings
\usepackage[T1]{fontenc}                                  % T1 font encoding
\usepackage{amsmath}                                      % AMS mathematical facilities
%\usepackage{amssymb}                                     % defines symbol names for math symbols in the fonts MSAM and MSBM
\usepackage[bitstream-charter]{mathdesign}                % mathematical fonts to fit with Bitstream Charter
\usepackage{courier}                                      % replaces the current typewriter font for Adobe Courier
\usepackage{listings}                                     % source code printer
\usepackage{color}                                        % adding colors
\usepackage{fancyhdr}                                     % extensive control of page headers and footers
\usepackage{booktabs}                                     % publication quality tables
\usepackage{xcolor}                                       % driver-independent color extensions   
\usepackage{tikz}                                         % creating graphics programmatically
\usepackage{float}                                        % improved interface for floating objects
\usepackage{subfig}                                       % figures broken into subfigures
\usepackage{algorithmic}                                  % a suite of tools for typesetting algorithms in pseudo-code.
\usepackage[linesnumbered,ruled,vlined]{algorithm2e}      % floating algorithm environment with algorithmic keywords

\definecolor{lightgray}{rgb}{.95,.95,.95}

\lstset{backgroundcolor=\color{lightgray},
        basicstyle=\ttfamily\fontsize{9pt}{9pt}\selectfont\upshape,
				commentstyle=\rmfamily\slshape,
				keywordstyle=\rmfamily\bfseries\color{black},
				captionpos=b,
				showstringspaces=false,
				breaklines=true,
				frame=lines,
				tabsize=2,
				aboveskip={\baselineskip}
}

\renewcommand{\labelenumi}{(\arabic{enumi})}
\renewcommand{\labelenumii}{(\alph{enumii})}

\subject{Anwendungsf�lle}
\title{shoptimize}
\author{Julian Fleischer $\cdot$ Konrad Reiche $\cdot$ S�ren Titze $\cdot$ Andr� Zoufahl}
\date{\today}
\publishers{\vskip 2ex InformatiCup 2012}

\begin{document}

\pagestyle{fancy}
\maketitle

\section{Beispieldaten laden}

\begin{tabular}{|c|p{0.7\textwidth}|}
\hline
\textbf{Akteure}                & Benutzer, GUI, System\\
\hline
\textbf{Anwendungsbereich}      & GUI\\
\hline
\textbf{Vorrausetzung}          & Benutzer hat die Seite aufgerufen\\
\hline
\textbf{Mindeszusicherung}      & Alle JavaScirpt Bibliotheken wurden erfolgreich geladen\\
\hline
\textbf{Zusicherung bei Erfolg} & Die ausgew�hlten Daten wurden analysiert, geladen und der Benutzer befindet sich im Kontrollbereich\\
\hline
\textbf{Haupterfolgsszenario}   & 1. Benutzer klickt auf ``Demo-Szenario laden''\\
                                & 2. Benutzer w�hlt eine der Demos aus\\
                                & 3. System validiert die geladene Demo durch Analyse der Daten\\
                                & 4. GUI zeigt den Kontrollbereich an.\\
\hline
\textbf{Erweiterungen}          & \\
\hline
\textbf{Implementierung}        & Die Demos liegen in programmatisch als Zeichenkette vor die wie eine normale Datei analyisiert werden.\\
\hline
\textbf{Implementiert?}         & Ja\\
\hline
\end{tabular}

\section{Optionen f�r Szenarien ausw�hlen}

\begin{tabular}{|c|p{0.7\textwidth}|}
\hline
\textbf{Akteure}                & Benutzer, GUI, System\\
\hline
\textbf{Anwendungsbereich}      & GUI\\
\hline
\textbf{Vorrausetzung}          & Benutzer hat die Seite aufgerufen\\
\hline
\textbf{Mindeszusicherung}      & Ein Datensatz wurde geladen und der Benutzer befindet sich im Kontrollbereich.\\
\hline
\textbf{Zusicherung bei Erfolg} & Das System hat die Konfiguration des Algorithmus ge�ndert.\\
\hline
\textbf{Haupterfolgsszenario}   & 1. Benutzer klickt auf ``Einstellungen''\\
                                & 2. Benutzer w�hlt die Option ``Wenige Gesch�fte'' aus\\
                                & 3. System liest die angepassten Daten aus\\
                                & 4. Benutzer klickt auf zur�ck\\
\hline
\textbf{Erweiterungen}          & 2a. Benutzer w�hlt die Option ``Keine Fahrtkosten'' aus\\
                                & 2b. Benutzer w�hlt die Option ``Hohe Fahrtkosten'' aus\\
\hline
\textbf{Implementierung}        & Die Szenarienoptionen werden durch die Parameter des Algorithmus umgesetzt.\\
\hline
\textbf{Implementiert?}         & Ja\\
\hline
\end{tabular}

\section{Individuelle Daten laden}

\begin{tabular}{|c|p{0.7\textwidth}|}
\hline
\textbf{Akteure}                & Benutzer, GUI, System\\
\hline
\textbf{Anwendungsbereich}      & GUI\\
\hline
\textbf{Vorrausetzung}          & Benutzer hat die Seite aufgerufen\\
\hline
\textbf{Mindeszusicherung}      & Alle JavaScirpt Bibliotheken wurden erfolgreich geladen\\
\hline
\textbf{Zusicherung bei Erfolg} & Die ausgew�hlten Daten wurden analysiert und geladen\\
\hline
\textbf{Haupterfolgsszenario}   & 1. Benutzer klickt auf ``Dateien laden''\\
                                & 2. Benutzer w�hlt die ``Fahrtkostendatei`` aus\\
                                & 3. System validiert die ''Fahrtkostendatei``\\
                                & 4. Benutzer w�hlt die ''Artikelpreisedatei`` aus\\
                                & 5. System validiert die ''Artikelpreisedatei``\\
                                & 6. GUI blendet eine Schlaffl�che ''Weiter`` ein\\
                                & 7. Benutzer klickt auf ''Weiter``\\
\hline
\textbf{Erweiterungen}          & 3a. System findet einen Fehler in der ''Fahrtkostendatei``\\
                                & 3b. GUI zeigt die Fehlermeldung an.\\
                                & 4a. System findet einen Fehler in der ''Artikelpreisedatei``\\
                                & 4b. GUI zeigt die Fehlermeldung an.\\
                                & 6a. Benutzer modifiziert die Daten im Eingabefeld.\\
                                & 7a. System validiert die modifizierten Daten aus dem Eingabefeld.\\
\hline
\textbf{Implementierung}        & Durch das Modul \emph{parser.coffee} implementiert, die die HTML File API verwendet\\
\hline
\textbf{Implementiert?}         & Ja\\
\hline
\end{tabular}

\section{Route finden}

\begin{tabular}{|c|p{0.7\textwidth}|}
\hline
\textbf{Akteure}                & Benutzer, GUI, System\\
\hline
\textbf{Anwendungsbereich}      & GUI\\
\hline
\textbf{Vorrausetzung}          & Benutzer hat die Seite aufgerufen\\
\hline
\textbf{Mindeszusicherung}      & Der Benutzer hat einen Datensatz geladen.\\
\hline
\textbf{Zusicherung bei Erfolg} & Der Benutzer befindet sich im L�sungsbereich und die beste Route wird angezeigt\\
\hline
\textbf{Haupterfolgsszenario}   & 1. Benutzer klickt auf ''Route finden``\\
                                & 2. System startet den Algorithmus\\
                                & 3. GUI blendet den L�sungsbereich ein\\
                                & 4. GUI visualisiert die L�sung auf zwei verschiedenen Graphen\\
                                & 5. GUI blendet eine Schlaffl�che ''Zur�ck`` ein\\
\hline
\textbf{Erweiterungen}          & 1a. Benutzer klickt auf ''Einstellungen``\\
                                & 1b. Benutzer konfiguriert den Algorithmus.\\
                                & 1c. Benutzer klickt auf ''Route finden``\\
\hline
\textbf{Implementierung}        & Der L�sungsbereich ist durch 4 Coffee Script Dateien implementiert: \emph{routePanel.coffee}, \emph{gmaps.coffee}, \emph{flot.coffee}, \emph{jit.coffee}\\
\hline
\textbf{Implementiert?}         & Ja\\
\hline
\end{tabular}

\section{M�gliche Einkaufsroute ausw�hlen}

\begin{tabular}{|c|p{0.7\textwidth}|}
\hline
\textbf{Akteure}                & Benutzer, GUI, System\\
\hline
\textbf{Anwendungsbereich}      & GUI\\
\hline
\textbf{Vorrausetzung}          & Benutzer hat die Seite aufgerufen\\
\hline
\textbf{Mindeszusicherung}      & Benutzer befindet sich im L�sungsbereich.\\
\hline
\textbf{Zusicherung bei Erfolg} & Der Benutzer befindet sich im L�sungsbereich und die beste Route wird angezeigt\\
\hline
\textbf{Haupterfolgsszenario}   & 1. Benutzer f�hrt mit der Maus �ber den oberen Graphen\\
                                & 2. GUI blendet f�r jedes Iterationsergebnis in der N�he der Maus ein Fadenkreuz ein.\\
                                & 3. GUI markiert im unteren Graphen die vorgeschlagene Route.\\
                                & 3. Benutzer klickt auf das Fadenkreuz\\
                                & 4. GUI speichert die Markierung im unteren Graphen.\\
                                & 5. GUI blendet an der Seite die Details der Route ein.\\
                                & 6. Benutzer bewegt die Maus �ber ein Detail der Route im linken Feld.\\
                                & 7. GUI markiert im unteren Graphen den Teilabschnitt der Route.\\
\hline
\textbf{Erweiterungen}          & \\
\hline
\textbf{Implementierung}        & F�r den unteren Graphen wurde \emph{JavaScript InfoVis Toolkit (JIT)} verwendet, f�r den oberen Graphen \emph{Flot}. Das Markieren der Route war urspr�nglich nicht m�glich. Dies wurde durch Implementierung von individuellen Kanten im Graph realisiert.\\
\hline
\textbf{Implementiert?}         & Ja\\
\hline
\end{tabular}

\section{Algorithmus erkl�ren lassen}

\begin{tabular}{|c|p{0.7\textwidth}|}
\hline
\textbf{Akteure}                & Benutzer, GUI, System\\
\hline
\textbf{Anwendungsbereich}      & GUI\\
\hline
\textbf{Vorrausetzung}          & Benutzer hat die Seite aufgerufen\\
\hline
\textbf{Mindeszusicherung}      & Alle JavaScirpt Bibliotheken wurden erfolgreich geladen\\
\hline
\textbf{Zusicherung bei Erfolg} & Der Benutzer hat den Algorithmus im groben verstanden.\\
\hline
\textbf{Haupterfolgsszenario}   & 1. Benutzer klickt auf ''Algorithmus erkl�ren``\\
                                & 2. System w�hlt ein Demobeispiel aus\\
                                & 3. System analyisiert die Daten der Demo\\
                                & 4. GUI blendet Animation ein die den Algorithmus erkl�rt\\
\hline
\textbf{Erweiterungen}          & \\
\hline
\textbf{Implementierung}        & Die Erkl�rung des Algorithmus ist dynamisch, d.h. sie funktioniert auf verschiedenen Eingabedaten, jedoch nur bis maximal 5 Gesch�fte und 10 Artikel.\\
\hline
\textbf{Implementiert?}         & Ja\\
\hline
\end{tabular}

\end{document}